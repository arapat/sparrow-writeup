%%%%%%%% SysML 2019 EXAMPLE LATEX SUBMISSION FILE %%%%%%%%%%%%%%%%%

\documentclass{article}

% Recommended, but optional, packages for figures and better typesetting:
\usepackage{microtype}
\usepackage{graphicx}
\usepackage{subfigure}
\usepackage{booktabs} % for professional tables
\usepackage{bbm}
% authors added following packages
\usepackage{multirow}

% hyperref makes hyperlinks in the resulting PDF.
% If your build breaks (sometimes temporarily if a hyperlink spans a page)
% please comment out the following usepackage line and replace
% \usepackage{sysml2019} with \usepackage[nohyperref]{sysml2019} above.
\usepackage{hyperref}

% Attempt to make hyperref and algorithmic work together better:
\newcommand{\theHalgorithm}{\arabic{algorithm}}

% Use the following line for the initial blind version submitted for review:
\usepackage{sysml2019}

% If accepted, instead use the following line for the camera-ready submission:
%\usepackage[accepted]{sysml2019}

% The \sysmltitle you define below is probably too long as a header.
% Therefore, a short form for the running title is supplied here:
\sysmltitlerunning{Submission and Formatting Instructions for SysML 2019}

\graphicspath{{figs/}}

% commands for commenting
\newcommand{\sidecomment}[2]{\marginpar{\footnotesize{\textcolor{#1}{#2}}}}
\newcommand{\janote}[1]{\sidecomment{red}{#1}}
\newcommand{\yonote}[1]{\sidecomment{blue}{#1}}

% all other commands
\newtheorem{theorem}{Theorem}
\newtheorem{corollary}{Corollary}
\newtheorem{definition}{Definition}


\newcommand{\loss}{\ell}
\newcommand{\eloss}{\widehat{\ell}}
\newcommand{\vx}{\vec{x}}
\newcommand{\Prob}[2]{P_{#1}\left[#2 \right]}
\newcommand{\cH}{\mathcal{W}}
\newcommand{\cX}{\mathcal{X}}
\newcommand{\cY}{\mathcal{Y}}
\newcommand{\cL}{\mathcal{L}}
\newcommand{\binary}{\{-1, 1\}}
\newcommand{\bR}{\mathbb{R}}
\newcommand{\adaboost}{AdaBoost}
\newcommand{\adtree}{ADTree}
\newcommand{\stumpboost}{StumpBoost}
\newcommand{\err}{\mbox{err}}
\newcommand{\errub}{\widehat{\mbox{err}}}
\newcommand{\edge}{\gamma}
\newcommand{\edgeEmp}{\hat{\edge}}
\newcommand{\sign}{\mbox{sign}}
\newcommand{\bx}{{\bf x}}
\newcommand{\Dist}{{\cal D}}
\newcommand{\var}{\mbox{Var}}
\newcommand{\std}{\mbox{std}}
\newcommand{\training}{{\cal S}}
\newcommand{\Z}{Z_{\training}}
\newcommand{\Ztest}{{\mathbf Z}}
\newcommand{\neff}{{n_{\mbox{eff}}}}
\newcommand{\Pre}{{\cal P}}
\newcommand{\cond}{{\cal C}}
\newcommand{\Rule}{{\cal R}}
\newcommand{\D}{{\cal D}}
\newcommand{\margin}{{\mathrm margin}}
\newcommand{\true}{{\bf T}}
\newcommand{\Models}{{\cal M}}
\renewcommand{\H}{{\cal H}}
\renewcommand{\S}{{\bf S}}

\newcommand{\tmsn}{{\bf TMSN}}
\newcommand{\Sparrow}{{\bf Sparrow}}

\newcommand{\eff}{\mbox{ess}}
\newcommand{\wsum}{\sum_{i=1}^n w_i}
\newcommand{\wsqSum}{\sum_{i=1}^n w_i^2}

\newcommand{\weakRules}{{\cal W}}
\newcommand{\strongRules}{{\cal H}}

\begin{document}

\twocolumn[
\sysmltitle{Tell me something new \\ {\large a new appoach to parallel
  learning and its application to boosting}}

% It is OKAY to include author information, even for blind
% submissions: the style file will automatically remove it for you
% unless you've provided the [accepted] option to the sysml2019
% package.

% List of affiliations: The first argument should be a (short)
% identifier you will use later to specify author affiliations
% Academic affiliations should list Department, University, City, Region, Country
% Industry affiliations should list Company, City, Region, Country

% You can specify symbols, otherwise they are numbered in order.
% Ideally, you should not use this facility. Affiliations will be numbered
% in order of appearance and this is the preferred way.
\sysmlsetsymbol{equal}{*}

\begin{sysmlauthorlist}
\sysmlauthor{Julaiti Alafate}{ucsd}
\sysmlauthor{Yoav Freund}{ucsd}
\end{sysmlauthorlist}

\sysmlaffiliation{ucsd}{
Department of Computer Science and Engineering,
University of California, San Diego,
La Jolla, CA, USA}

\sysmlcorrespondingauthor{Julaiti Alafate}{jalafate@eng.ucsd.edu}
\sysmlcorrespondingauthor{Yoav Freund}{yfreund@ucsd.edu}

% You may provide any keywords that you
% find helpful for describing your paper; these are used to populate
% the "keywords" metadata in the PDF but will not be shown in the document
% \sysmlkeywords{Machine Learning, SysML}

\vskip 0.3in

\begin{abstract}
  {\bf Rewrite abstract}
  
This paper presents several methods for reducing the I/O
and communication requirements of parallelized boosting.

The inner loop of the tree boosting algorithm is a search for
splitting rules. This search is driven by estimates of expected loss.
 Traditionally, these estimates were based on {\em all
   available data}. Previous work~\cite{} has shown that early
 stopping can be used to reduce the amount of data required.

 In this work we extend the work on early stopping in two ways, both
 of which further increase the speed of the boosting algorithm. First,
 we show how early stopping can be used to parallelize the boosting
 algorithm. Second, we show how memory pressure can be reduced in
 when the examples have non-uniform weights.
 
We describe these improvement and show their mathematical properties.
We implemented our methods for boosting and evaluated them on the
splice-site prediction problem~\cite{sonnenburg_coffin_2010,
  agarwal_reliable_2014}. We compare our performance to that of 
SparkML~\cite{}
XGBoost~\cite{chen_xgboost:_2016},
LightGBM~\cite{ke_lightgbm:_2017}. Our implementation is about ten
times faster than the alternatives (fill in when we have details).
\end{abstract}
]

% this must go after the closing bracket ] following \twocolumn[ ...

% This command actually creates the footnote in the first column
% listing the affiliations and the copyright notice.
% The command takes one argument, which is text to display at the start of the footnote.
% The \sysmlEqualContribution command is standard text for equal contribution.
% Remove it (just {}) if you do not need this facility.

%\printAffiliationsAndNotice{}  % leave blank if no need to mention equal contribution
\printAffiliationsAndNotice{\sysmlEqualContribution} % otherwise use the standard text.


\section{Introduction}\label{sec:intro}

It is well known that I/O is the main bottleneck to the application of
machine learning algorithms to very large datasets. Improvements to
date organization on disk, memory and cache for ML algorithms are
active areas of research.(add refs here).

Many machine learning algorithms, including XGBoost,
follow the a similar pattern. At each iteration the complete training
set is scanned. Then, based on statistics calculated from the training
set, the model is updated to decrease the loss.

Scanning all of the data is obviously optimal from the point of view
of the statistical estimates. However, the incremental improvemen in
the model, given an increase from a fracton of the data to the full
data, might be small. If it is sufficiently small, then stopping early
and proceeding to the next iteration can result in significantly
faster running time.

This idea was introduced by Wald\cite{wald_sequential_1973} in the
1940s under the title ``Sequential analysis'' (SA). We describe SA
precisely in Section~\ref{sec:sequential-analysis}. As an illustrative
example, suppose that our goal is to find a classifier from
$M_1,\ldots,M_k$ whose error rate is close to the minimum across that
$k$ models. The standard approach is to scan all of the training
examples, compute the average error of each model, and choose the
model with the minimal error. The SA approach is to read the examples
one by one, each time updating the error estimate for each model, and
emply a specifically designed {\em stopping rule} to decide when to
stop and which model to output. An example where SA will stop much
before the standard approach is when one rule has an error rate of
$0.1$ while all of the other rules have error of $0.5$.

The use of stopping rules in boosting algorithms is not new, it was
studied before by Domingo and Watanabe~\cite{domingo_scaling_2000} and
by Bradley and Schapire~\cite{bradley_filterboost:_2007}.

Our main contribution in this paper a new parallel boosting algorithm
that takes advantage of early stopping. Observe that the main
computation step of the boosting algorithm is the search for a rule
that is slightly better than random guessing.~\footnote{Often the
  stated goal is to find the {\em best} weak rule. However, for the
  theory behind adaboost to hold, any rule whose error is smaller than
  $1/2$ will do.}

Both XGBoost and LightGBM are based on the
Bulk-Synchronization~\ref{valiant_bridging_1990} model of parallel
computation. In this model the workers are all synchronized at the
start of each boosting iteration. This establishes a well defined {\em
  common state} at the synchronization boundary which simplifies the
reasoning about the algorithm an it's interaction with the hardware
and the OS.

In our approach, which we call ``tell me something new'' or
\tmsn\ {\em we do away with both synchronization and common state}.

We first give a rough description of \tmsn\, followed with a more
refined one. Roughly, each of the worker is streaming over it's local
data to evaluate the error of a set of weak rules. It uses a stopping
rule to stop when it identifies a weak rule whose error is smaller
than $1/2-\gamma$. when this good weak rule has been identified, it
interrupts the search, adds the found rule to the strong rule, and
broadcasts this strong rule. The other workers, upon recieving the new
strong rule, interrupt their own search and use the new rule.

This rough version of \tmsn\ would work if the workers were
synchronized and communication was instantanous. Both are unrealistic
assumptions. To remove the need for these assumptions we add a {\em
  global measure of progress}, which is an upper bound on the expected
loss of the strong hypothesis. When a worker broadcasts a new strong
rule, it pairs it with this upper bound. When a worker recieves a
strong rule, it acccepts it only if the newcomers upper bound is
smaller than the current upper bound. The full details are in
section~\ref{sec:}

We call our protocol ``tell me something new'' because the workers
send out information only when they have ``something new'' which in
our case means a significantly better strong rule. This protocol has
several desirable properties:
\begin{enumerate}
\item {\bf No head-node} Most distributed systems rely on a head-node
  that synchronizes the workers. The head node is a
  single-point-of-failure and a bottleneck, especially systems with a
  large number of workers. \tmsn\ avoids these problems because it
  does not require a head-node.
\item {\bf Reduced communication bandwidth:} In typical bulk-synchronous
  protocols all workers send and recieve information from the head
  node at each step. This can mean a lot of communication even when
  there is little progress in terms of reducing loss. The workers in
  \tmsn\ remain mute when as long as they don't find anything useful.
\item {\bf No blocking} The communication operations are all
  non-blocking, in other words, at no point does a worker wait for a
  response from another worker. This means that CPU utilzation is not
  reduce by wait times.
\item {\bf Robustness} As a result of the fact that there is no
  synchronization and no head node, they system is very robust. If a
  single computer is slow or crashes, the effect on the other
  computers is small. Computers can be removed or added at any time
  and will quickly catch up to the current best model and start
  contributing their cycles to the search.
\end{enumerate}

\iffalse
A data structure $M$ defines the model, the
algorithm compares several alternative models $M_1,\ldots,M_k$. Ideally,
models should be compared by expected loss, defined as 
$$\loss(M) = E_{\vx \sim \Dist}(L(M,\vx))$$
where $\vx$ is an example chosen independently at random according to
the distribution $\Dist$ and $L(M,\vx)$ is a loss function which associates a
real number with the application of a model $M$ to the example $\vx$.
For reasons that will be revealed shortly, we assume that the range of
$L$ is $[0,1]$.

We refer to the expected loss $\loss(M)$ as the {\em ideal} because in
order to calculate it we need an infinite number of example. As in
practice the number of examples is finite, the algorithm has to use an {\em estimate}.

We denote a model by $M$ and a data point by $\vx$. The datapoints are
generated by a fixed but unknown distribution $\Dist$. A loss function $L$
assigns a real valued to a model $M$ applied to a datapoint $\vx$:
$L(M,\vx)$. We restrict our losses to the values $[0,1]$.

The quantity that we wish to estimate is the expected loss with
respect to the true distribution $\Dist$: $\loss(M) = E_{\vx \sim
  \Dist}(L(M,\vx))$. We cannot caculate $\loss(M)$, we can only
{\em estimate} it from our finite traning data: $T = \{ \vx_1, \ldots,
\vx_n\}$. The most common estimator is the unweighted average:
$$ \eloss_n(M) = \frac{1}{n} \sum_{i=1}^n L(M,\vx_i)$$




Most of this work ignores a fundamental charasteristic
of machine learning tasks - the bias-variance tradeoff. Most machine
learning algorithm consider only the {\em mean} of the loss

On the other hand, computer clock rates are unlikely to increase
beyond 4\,GHz in the foreseeable future.  As a result there is a keen
interest in parallelized machine
learning algorithms~\cite{bekkerman_scaling_2012}.

The most common approach to parallel ML is based on Valiant's bulk
synchronous~\cite{valiant_bridging_1990} model. This approach calls for a
set of workers and a master. The system works in (bulk) iterations. In each iteration
the master sends a task to each worker and then waits for its
response. Once {\em all} machines responded, the master proceeds to the
next iteration. Thus the head node enforces synchronization (at the
iteration level) and maintains
a state that is shared by all of the workers.

Unfortunately, bulk synchronization does not scale well to more than
10--20 computers. Network congestion, latencies due to synchronization,
laggards, and failing computers result in diminishing benefits from
adding more workers to the
cluster~\cite{zaharia_apache_2016,mcsherry_scalability!_2015}.

There have been several attempts to break out of the bulk-synchronized
framework, most notably the work of Recht et~al.\ on
Hogwild~\cite{recht_hogwild:_2011} and Lian et~al.\ on asynchronous
stochastic descent~\cite{lian_asynchronous_2015}. Hogwild
significantly reduces the synchronization penalty by using
asynchronous updates and parameter servers. The basic idea is to
decentralize the task of maintaining a global state and relying on
sparse updates to limit the frequency of update clashes.

\paragraph{Tell Me Something New}

Our first contribution is a new approach for parallelizing ML algorithms
which eliminates synchronization and the global state and instead uses a
distributed policy that guarantees progress. We call this approach
``Tell Me Something New'' (\tmsn). To explain \tmsn\ we start with an
analogy.

Consider a team of a hundred investigators that is going through
thousands of documents to build a criminal case where time is of the
issue. Assume also that most of the documents contain little or no new
information. How should the investigators communicate their findings
with each other? We contrast the bulk-synchronous (BS) approach and the
\tmsn\ approach. In the BS approach, each investigator takes a stack
of documents to their cubicle and reads through it. Then all of the
investigator meet in a room and tell each other what they found. Once
they are done, the process repeats. One problem with this approach is
that the fast readers have to wait for the slow readers. Another is
that a decision needs to be made as to how many documents or pages, to
put in each stack. Too many and the iterations would be very slow, too
few and all of the time would be spent in meetings.

The \tmsn\ approach is radically different. In this approach, each
investigator gets documents independently according to their speed of
reading and work habits. There is no meeting either. Instead, when
an investigator finds a piece of information that she believes is new,
she stands up in her cubicle and tells all of the other workers about
it. This has several advantages: nobody is ever waiting for anybody
else; the new information is broadcasted as soon as it is available, and
the system is fault resilient --- somebody falling asleep has little
effect on the others.
The analogy to parallel ML maps investigators
to computers, ``case'' to ``model'', and ``new information'' to
``improved model''.

%\iffalse
Consider a team of investigators that is going through
thousands of documents to build a criminal case. Assume also that most
of the documents contain no new information. It makes little sense for
them to send each other a summary of each document they read, they
would just be wasting each other's time that way. What would make more
sense is for each to sit reading in their cubicle until one of them
identifies a document with new information, that person than stands
up, interrupts all of the other people, and gives each of them a
summary of what they found. This ensures that all of the investigators
are updated as soon as new information is available, but are otherwise
left to do their work.
%\fi

More concretely, \tmsn\ for model learning works as follows. Each
worker has a model $H$ and an upper bound $L$ on the true loss of
$H$. The worker searches for a better model $H'$ whose loss upper
bound is $L'$. If $L'$ is significantly smaller than $L$, then the
worker takes two actions. First, $H',L'$ replaces $H,L$. Second
$(H',L')$ is broadcast to all other workers.  Each worker also listens
to the broadcast channel. If it receives pair $(H',L')$ it checks
whether $L'$ is significantly lower than its own upper bound $L$. If
it is, the worker replaces $(H,L)$ with $(H',L')$. Otherwise, the worker discards the
pair.

\paragraph{Boosting trees using TMSN}
Our second contribution is an application of \tmsn\ to boosted
decision trees. Boosted trees is a highly effective and widely used
machine learning method. In recent years there have been several
implementations of boosting that greatly improve over previous
implementations in terms of running time, in particular,
XGBoost~\cite{chen_xgboost:_2016} and
LightGBM~\cite{ke_lightgbm:_2017}. These implementations scale up to
training sets of many millions, or even billions of training examples.
Both implementations can run in one of two configurations: a
memory-only configuration where all of the training data is stored in
main memory, and a memory and disk configuration where the data is on
disk and is copied into memory when needed. The memory-only version is
significantly faster, but require a machine with very large
memory.

We present an implementation of boosting tree learning using
\tmsn\ that we call \Sparrow. This is a disk and memory
implementation, which requires only a fraction of the training data to
be stored in memory.  Yet, as our comparative experiments show, it is
about 10 times faster than XGBoost and LightGBM using the {\em memory
  only} configuration.

The rest of the paper is divided into four sections.
First we give a general description of \tmsn\ in Section~\ref{sec:tmsn}.
Then we introduce a special application of our algorithm, namely \Sparrow, in Section~\ref{sec:boost}.
After that we describe in more details of the algorithms and the system design of \Sparrow\ in
Section~\ref{sec:Algorithms}.
Finally, we present empirical results in Section~\ref{sec:experiments}.
\fi

\section{Tell Me Something New}\label{sec:tmsn}
We start with a general description of \tmsn\ which will be followed
by a description of \tmsn\ for boosting. To streamline our presentation
we consider binary classification, but other supervised or
unsupervised learning problem can be accommodated with little change.

We are given
\newcommand{\cD}{{\cal D}}
\begin{itemize}
\item A set of classifiers $\strongRules$, each classifier $H \in
  \strongRules$ is a mapping from an input space $X$ to a binary label $\{-1,+1\}$.
\item A stream of labeled examples $(x_1,y_1),(x_2,y_2),\ldots$, $x_i
  \in X$, $y_i \in \{-1,+1\}$, generated IID according to a fixed but
  unknown distribution $\cD$.
\end{itemize}

The goal of the algorithm is to find a classifier $H \in
\strongRules$ that minimized the error probability $\err(H)\doteq
P_{(x,y) \sim \cD}[H(x) \neq y]$

All workers start from the same initial classifier $H_0$ which is
improved iteratively. Some iterations end with the worker finding a
better classifier by itself, others end with the worker receiving a
better classifier from another worker. The sequences of classifiers
corresponding to different workers can be different, but with high
probability they all converge to the same classifier.

Denote each worker by an index $i=1,\ldots,n$. On iteration $t$
each worker has its {\em current} classifier  $H_i(t)$ and a set of $m$
{\em candidate} classifiers $G_i^j(t)$. An error upper bound
$\errub(H_i(t))$ is associated with $H_i(t)$ so that with high
probability $\errub(H_i(t)) \geq \err(H_i(t))$.

\begin{figure}[t]
%\begin{minipage}{.31\textwidth}
\begin{center}
  \includegraphics[width=0.7\textwidth]{AsyncUpdates.pdf}
\end{center}
  \caption{{\bf Execution timeline of a \tmsn\ system}
      System consists of four workers. The first update occurs when
      worker 3 identifies a better classifier $H_1$. It then replaces
      $H_0$ with $H_1$ and broadcasts $(H_1,z_1)$ to the
    other workers. The other workers receive the message the at different
    times, depending on network congestion. At that time they  interrupt the
    scanner (yellow explosions) and start using $H_1$. Next, worker 2
    identifies an improved rule $H_2$ and the same process ensues.
    \label{fig:async}}
   	\vspace{0pt}
%\end{minipage}
\end{figure}

The worker reads examples from the stream and uses them to estimate
the errors of the candidates. It stops when it finds a candidate that,
with high probability, has an error smaller than
$\errub(H_i(t))-\epsilon$ for some constant ``gap'' parameter
$\epsilon>0$.

More precisely, the worker uses a {\em stopping rule} that chooses a
stopping time and a candidate rule and has the property that, with
high probability, the chosen candidate rule has an error smaller than
$\errub(H_i(t))-\epsilon$. This candidate then replaces the current
classifier, the new upper bound is set to be $\errub(H_i(t+1)) =
\errub(H_i(t))-\epsilon$, a new set of candidates is chosen and the
worker proceeds to the next iteration. At the same time the worker
{\em broadcasts} the pair $(H_i(t+1), \errub(H_i(t+1))$.

A separate process in each worker listens to broadcasts of this
type. When worker $i$ receives a pair $(H,\errub(H))$ it compares the
upper bound $\errub(H)$ with the upper bound associated with it's
current classifier $\errub(h_i(t))$. If $\errub(H) < \errub(h_i(t))-\epsilon$,
it interrupts the current search and sets $H_i(t+1)=H$. If not the
received pair is discarded.

Note that the only assumption that the workers make regarding the
incoming messages is that the upper bound $\errub(H)$ is sound. In
other words that, with hight probability, it is an upper bound on the
true error $\err(H)$. There is no synchronization and if a worker is
slow or fails, the effect on the other workers is minimal.

Different implementations of \tmsn\ differ in the way that they
generate candidate classifiers and in the stopping rules that they
use. For \tmsn\ to be effective, the stopping rule should be both
sound and tight. If it is not sound, then the scheme falls apart, and
if it is not tight, then the stopping rules stop later than needed,
slowing down convergence.

Next, we describe how \tmsn\ is applied to boosting.

\section{Theory}\label{sec:theory}
We start with a brief description of the confidence-rated boosting
algorithm (Algorithm 9.1 on the page 274 of \cite{schapire_boosting:_2012}).

Let $\vx \in X$ be the feature vectors and let the output be $y \in
Y= \{-1,+1\}$. The target is defined as a joint distribution $\Dist$ over
$X \times Y$, our goal is to find a classifier $c: X \to Y$ with small
error:
$$\err_{\Dist}(c) \doteq \Prob{(\vx,y)\sim \Dist}{c(\vx) \neq y}$$

We are given a set $\cH$ of base classifiers $h:X \to [-1,+1]$. The
Score function generated by AdaBoost is a weighted sum of $T$ rules from
$\cH$
\[
S_T(\vx) = \left( \sum_{t=1}^T \alpha_t h_t(\vx) \right)
\]

The strong classifier is the sign of the score function: $H_T =
\sign(S_T)$.  

AdaBoost is a gradient descent algorithm. It operates by iteratively
decreasing the value of an exponential potential function\footnote{Other
  potential functions have been studied. In this work we restrict
  ourselves to the original exponential potential function.}  We start
with the {\em true} potential, i.e. the potential in the limit where
the size of the training set increases to infinity. We will then
consider the realistic case, where the size of the training set is
finite.  The {\em true} potential of the score function $S_t$ is
\[
\Phi(S_t) = \Expect{(\vx,y) \sim \Dist}{e^{-S_t(\vx)y}}
\]
Consider adding a single base rule $h_t$ to the score function
$S_{t-1}$:
$S_t=S_{t-1}+\alpha_t h_t$ and taking the partial derivative of the potential with respect
to $\alpha_t$ we get:
\begin{equation} \label{eqn:weights}
\left. \frac{\partial}{\partial \alpha_t}\right|_{\alpha_t=0} \Phi(S_{t-1}+\alpha_t h)
= \Expect{(\vx,y) \sim \Dist_{t-1}}{h(\vx) y}
\end{equation}
Where
\begin{equation} ~\label{eqn:Dt}
\Dist_{t-1} = \frac{\Dist}{Z_{t-1}} \exp\left( -S_{t-1}(\vx)y \right)
\end{equation}
and $Z_{t-1}$ is a normalization factor that makes $\Dist_{t-1}$ a
distribution.

AdaBoost performs coordinate-wise gradient descent where each
coordinate corresponds to one base rule. Using
equation~\ref{eqn:weights} we can express the gradient with respect to
base rule $h$ as a correlation, which we also call the {\em true edge}: 
\begin{equation} \label{eqn:true-edge}
\edge_{t}(h) \doteq \corr_{\Dist_{t-1}}(h) \doteq \Expect{(\vx,y) \sim \Dist_{t-1}}{h(\vx) y}
\end{equation}

The goal of a single boosting step is to find a base rule with
significant edge.~\footnote{Some times the goal is stated as
  finding the rule with the {\em maximal} edge.  Here we require only
  that $h_t$ has a significant edge.}

Up to this point, our discussion regarded the true expected value or
an infinitely large training set. Real training sets are finite, 
therefore \Sparrow\ has to {\em
  estimate} the edge of $h$. We use the standard unbiased estimate
\begin{equation} \label{eqn:emp-edge}
\edgeEmp_{t}(h) \doteq \corrEmp_{\Dist_{t-1}}(h)
\doteq 
\sum_{i=1}^n \frac{w_i}{Z_{t-1}} h(\vx_i) y_i
\end{equation}
where  $w_i = e^{-S_{t-1}(\vx_i)}$ and $Z_{t-1} = \sum_{i=1}^n w_i$

The novelty of \Sparrow\ is in the way it uses samples of the training
data to to identify rules whose true edge is significant.
Several statistical techniques are used to minimize the number of
examples needed to compute the estimates.

\subsection{Effective Sample Size}
\label{sec:effectiveSampleSize}
Equation~\ref{eqn:emp-edge} defines $\edgeEmp(h)$, which is an
unbiased estimate of $\edge(h)$. How accurate is this estimate? A
standard quantifier is the variance of the estimator:
\begin{equation} \label{eqn:variance}
 \mbox{Var}(\edgeEmp) = \frac{\sum_{i=1}^n w_i^2}{\left(\sum_{i=1}^n w_i\right)^2}
\end{equation}
If all of the weights are equal $\mbox{Var}(\edgeEmp) = 1/n$ which
corresponds to a standard deviation of $1/\sqrt{n}$ which is the
expected relation between the sample size and the error.

If the weights are not equal then the variance is larger and the
estimate is less accurate. We define the {\em effective sample size}
$\neff$ to be
\begin{equation} \label{eqn:neff}
  \neff \doteq \frac{\left(\sum_{i=1}^n w_i\right)^2}{\sum_{i=1}^n w_i^2}
\end{equation}
So that $\mbox{Var}(\edgeEmp) = 1/\neff$.

To see that the name ``effective sample size'' makes sense, consider
$n$ weights where $w_1=\cdots,w_k=1/k$ and
$w_{k+1}=\cdots=w_{n}=0$. It is easy to verify that in this case
$\neff=k$ which agrees with our intuition that examples with zero
weight have no effect on the estimate.

\Sparrow\ samples examples from disk to memory using {\em weighted
  sampling}. This is a sequential algorithm that reads from disk one
example $(\vx,y)$ at a time, calculates the weight for that example
$w_i$ and the flips a biased coin to accept the example with
probability $w_i/M$, where $M\geq \max w_i$. Accepted examples are
stored in main memory with initial weight of $1$.

LightGBM also uses sampling, but its sampling method is biased.

Suppose $n$ initial memory-resident training examples consist of 99\%
negative and 1\% positive examples. In that case the first base rule
will be ``predict negative everywhere''. After reweighting the total
weights of the positive and negative examples will be equal. This in
turn means that the the effective size of about $\neff \approx
n/25$. In other words, the variance of estimates of the edge has
increased by a factor of 25 relative to the initial uniform weights.

When \Sparrow\ detects that $\neff$ is small. It clears the memory and
samples a new set from disk. However, as the sampling uses the weights
as acceptance probabilities, half of the training data will be
positive and half of them will be negative and $\neff$ is back to $n$.

This process continues as long as AdaBoost is making progress and the
weights are becoming increasingly skewed. When the skew is large,
$\neff$ is small and \Sparrow\ resamples a new sample with uniform
weights.

The net effect is that even though the memory-resident samples are
small, they contain the important examples, whose weight is large.


\subsection{Sequential analysis}

\Sparrow\ achieves Disk-to-Memory efficiency by using weighted
resampling that is triggered when $\neff$ is too small.

\Sparrow\ achieve Memory-to-CPU efficiency by reading from memory the
minimal number of examples that are necessary to establish that a
particular weak rule has a significant edge. This is done using
Sequential Analysis and Early Stopping.

Sequential analysis (SA) was introduced by
Wald~\cite{wald_sequential_1973} in the 1940s.  Here we give a short
illustration. Suppose we want to estimate the expected loss of a
model. In the standard large deviation analysis we assume that the
loss is bounded in some range, say $[-M,+M]$ and that the size of the
training set is $n$. This implies that the standard deviation of the
training loss is at most $M/\sqrt{n}$. In order to make this standard
deviation be smaller than some $\epsilon>0$ we need that $n >
(M/\epsilon)^2$. While this analysis is optimal in the worst case, it
can be improved if we have additional information about the standard
deviation. We can glean such information from the observed losses by
using the following sequential analysis method. Instead of choosing
$n$ ahead of time, the algorithm computes the loss of one example at a
time. A {\em stopping rule} is used to decide whether, conditioned on
the sequence of losses seen so far, there is very small probability
that the difference between the average loss and the true loss is
larger than $\epsilon>0$. The result is that when the standard
deviation is significantly smaller than $M$ the number of examples
that need to be used in the estimate is much smaller than
$n=(M/\epsilon)^2$.

\subsection{\Sparrow's stopping rule} \label{sec:balsubramani}

Using stopping rules for AdaBoost was proposed in
~\cite{domingo_scaling_2000, bradley_filterboost:_2007}. Each use a
different stopping rule. Unlike those works, we are interested in a
rule that will take into account the fact that examples have different
weights which impacts the the variance of the estimator. We want a
stopping rule that is sensitive to $\neff$.

Note that the choice of stopping rule has a direct impact on the
performance of the algorithm. In other words, we want to use a
stopping rule that is as tight as possible, including constants.

We use a stopping rule proposed in~\cite{balsubramani_sharp_2014}
for which they prove the following:

\begin{theorem}[based on \cite{balsubramani_sharp_2014} Theorem 4] \label{thm:balsubramani}
  Let $M_t$ be a martingale $M_t = \sum_i^t X_i$,
  and suppose there are constants $\{c_k\}_{k \geq 1}$ such that
  for all $i \geq 1$, $|X_i| \leq c_i$ w.p.\ 1.
  For $\forall \sigma > 0$, with probability at least $1 - \sigma$ we have
  \[
  \forall t: |M_t| \leq C \sqrt{
    \left( \sum_{i=1}^t c_i^2 \right)
    \left( \log \log \left( \frac{ \sum_{i=1}^t c_i^2 }{ |M_t| }\right) +
    \log \frac{1}{\sigma} \right)
  },
  \]
  where $C$ is a universal constant.
\end{theorem}

As we care about constants, we did a series of experiments to find the
best constant $C$ which we use in \Sparrow.

\section{System Design and Algorithms} \label{sec:Algorithms}

\begin{figure}
\centering
    \includegraphics[width=0.5\textwidth]{figs/HeaderScanner.png}
    \caption{The \Sparrow\ system architecture.}\label{fig:architecture}
    \vspace{0pt}
%\end{minipage}
\end{figure}

In this section we describe \Sparrow.

The main procedure of \Sparrow\ is for identifying a rule that has a significant
edge (Equation~\ref{eqn:true-edge}). There are two
subroutines that can execute in parallel in the process: a
{\bf Header } and multiple {\bf Scanners}. We describe each subroutine in turn.


\subsection*{Scanner}

The task of a scanner (element {\bf (d)} in Figure~\ref{fig:architecture})
is to read training examples sequentially and stop
when it has identified one of the rules to be a {\em good} rule.
It is is initiated by the Header given 3 parameters: (1) current
strong rule $H_t$, (2) target edge $\gamma_t$ which is the threshold
that defines a {\em good} rule, and (3) a sample from the training data.

More specifically, at any time point the scanner stores the current strong
rule $H_t$, a set of candidate weak rules $\weakRules$ (which
define the candidate strong rules), and a target
edge $\gamma_t$. The scanner scans the training examples stored in
memory sequentially, one at a time. It computes the weight of the
examples using $H_t$ and then updates a running estimate of the edge
of each weak rule $h \in \weakRules$.

The scan stops when the stopping rule determine that
the true edge of a particular weak rule
$\gamma(h_t)$ is, with high probability,
larger than a threshold $\gamma$. The
worker then adds the identified weak rule $h_t$ {\bf (f)} to the current
strong rule $H_t$ to create a new strong rule $H_{t+1}$.
The weight of the added rule is calculated assuming that its edge is
equal to $\gamma$.

The scanner falls into the \textit{Failed} status if after exhausting
all examples in the current sample set, no weak rule with an advantage
larger than the threshold $\gamma$ is detected.

When a scanner terminates, it communicates either the newly accepted
weak rule or a \textit{Failed} message back to the Header.
The Header then updates the strong rule and the target edge accordingly.


\subsection*{Header}

The Header plays a rule similar to the Parameter Server. It consists
of two components, a Scheduler and a Sampler.


The Scheduler maintains a global model. It receives the model updates from scanners, and decides whether or not accepting the updates to the global model. If it accepts the model update, the Scheduler updates the global model accordingly, and broadcast the new global model to all Scanners. If the model updates is incompatible with the global model, for example, if it is based on an out-dated global model version, the Scheduler would reject the model updates.

The Scheduler also maintains the value of
the targeted edge $\gamma$. If no scanner finds a new model update for a long period of time, the Scheduler decreases the value of the targeted edge, and broadcast the new value.

Our assumption is that the entire training dataset does
not fit into main memory and is therefore stored in external storage
{\bf (a)}. As boosting progresses, the weights of the examples become
increasingly skewed, making the dataset in memory effectively smaller.
To counteract that skew, {\bf Sampler} prepares a {\em new}
training set, in which all of the examples have equal weight, by using
selective sampling. When the effective sample size associated
with the old training set becomes too small, the scanner stops using
the old training set and starts using the new one.\footnote{The
  sampler and scanner can run in parallel on separate cores. However in
  our current implementation the worker alternates between scanning and
  sampling.}


The Sampler uses selective sampling by which we mean that the
probability that an example $(x,y)$ is added to the sample is
proportional to $w(x,y)$. Each added example is assigned an initial
{\bf weight} of $1$.
{There are several known algorithms
  for selective sampling. The best known one is rejection sampling
  where a biased coin is flipped for each example. We use a method
  known as \textit{minimal variance sampling}~\cite{kitagawa_monte_1996}
  because it produces less variation in the sampled set.}
  
\paragraph*{Incremental Updates} Our experience shows that the most
time consuming part of our algorithms is the computation of the
predictions of the strong rules $H_t$. A natural way to reduce this
computation is to perform it incrementally. In our case this is
slightly more complex than in XGBoost or LightGBM, because {\bf
  Scanner} scans only a  fraction of the examples at each
iteration. To implement incremental update we store for each example,
whether it is on disk or in memory, the results of the latest
update. Specifically, we store for each training example the tuple
$(x, y, w_s, w_l,H_l)$, Where $x,y$ are the feature vector and the
label, $H_l$ is the strong rule last used to calculate the weight of
the example. $w_l$ is the weight last calculated, and $w_s$ is
example's weight when it was last sampled by the sampler. In this way
{\bf Scanner} and {\bf Sampler} share the burden of computing
the weights, a cost that turns out to be the lion's share of the total
run time for our system.




\section{Tuning a good policy to set $\gamma$}

The main challenge of Sparrow is to adjust the value of $\gamma$ at a right pace.

Current policy for adjusting gamma is not robust. The header node decreases
the value of gamma when a certain number of scanners failed to trigger stopping
rules.
If the values of gamma decrease too fast, we risk converging to a sub-optimal loss.
In Figure~\ref{fig:rapid-gamma}, the green curve corresponds to the setting
where the values of $\gamma$ decrease too fast.
The green curve (trained with 50 scanners) decreases faster than the orange curve (trained with 10 scanners).
The value of gamma decreases every time it sees a fixed number of failures, where a failure is defined as a packet from any scanner saying that the stopping rule failed to trigger after scanning all samples. With more scanners, the head node received more failure packets, and decreased the value of gamma more aggressively.

On the other hand, if the values of $\gamma$ decreases too conservatively,
it could take a very long time to find the right value of $\gamma$ which can trigger the stopping rule.

\begin{figure}
\centering
    \includegraphics[width=0.5\textwidth]{paper/figs/stale.png}
    \caption{The decrease of loss stales when we decreases the value of gamma too rapidly.}\label{fig:rapid-gamma}
    \vspace{0pt}
%\end{minipage}
\end{figure}

One other problem with asynchronous distributed training is that
with more scanners, we are more likely to find a wrong weak by chance.
In Figure~\ref{fig:overfit},
the green curve starts to perform worse on the test set as the result of overfitting
to the sample set from the training data.
With more scanners we are more likely to find a bad tree. If we then assign a large $\gamma$ to the
bad tree, the performance on the test suffers.

\begin{figure}
\centering
    \includegraphics[width=0.5\textwidth]{paper/figs/overfit.png}
    \caption{The weak rule could over-fit to the sampled data when we decreases the value of gamma too slowly.}\label{fig:overfit}
    \vspace{0pt}
%\end{minipage}
\end{figure}
\section{Experiments}\label{sec:experiments}

In this section we describe the results of experiments comparing
the run time of \Sparrow\ with those of two leading implementations of
boosted trees: XGBoost and LightGBM.


\begin{table}[]
\centering
\label{table-exp}
\begin{tabular}{|l|l|c|c|}
\hline
Memory       & \Sparrow         & XGBoost             & LightGBM       \\ \hline
16 GB        & 18.5             & 3224.7 (on disk) & (crash)        \\
32 GB        & 17.9             & 1705.8 (on disk) & (crash)        \\
72 GB        & 21.9             & 1497.7 (on disk) & 108.1          \\
144 GB       & 19.6             & 241.4 (in memory)   & 109.4          \\ \hline

% \end{tabular}
% \vspace{0.2cm}
% \begin{tabular}{|l|l|c|c|}
%                 & \Sparrow         & XGBoost             & LightGBM       \\ \hline
\#~Rules  & 189    & 400                 & 400            \\ \hline
\end{tabular}

\vspace{0.2cm}

\caption{Comparison of the total training time on
the splice site detection task (minutes) and the number
of rules in the final ensemble when it converges}
\end{table}


\subsection{Setup}

We use a large dataset that was used in other studies of large scale
learning on detecting human acceptor splice site~\cite{sonnenburg_coffin_2010, agarwal_reliable_2014}.
The learning task is binary classification.
We use the same training dataset of 50\,M samples as in the other work,
and validate the model on the testing data set of 4.6\,M samples.
The training dataset on disk takes over 27\,GB in size.

In current implementation of \Sparrow, we restrict our trees to one
level so-called ``decision stumps''. We plan to perform comparisons
using multi-level trees and more than two labels. We expect similar
runtime performance there. To generate comparable models,
we also train decision stumps in XGBoost and LightGBM
(by setting the maximum tree depth to 1).

Both XGBoost and LightGBM are highly optimized, and support multiple
tree construction algorithms.
For XGBoost, we chose approximate greedy algorithm which is its fastest method.
LightGBM supports using sampling in the training,
which they called \textit{Gradient-based One-Side Sampling} (GOSS).
GOSS keeps a fixed percentage of examples with large gradients,
and then randomly sample from remaining examples with small gradients.
We selected GOSS as the tree construction algorithm for LightGBM.

All algorithms in comparison optimize the exponential loss as defined in AdaBoost.
We also evaluated the final model by calculating its area under precision-recall
curve (AUPRC) on the testing dataset.

Finally, the experiments are all conducted on EC2 instances from Amazon Web Services.
We ran the evaluations on four different instance types with increasing memory capacities,
specifically
16\,GB (\texttt{c5d.2xlarge}), 32\,GB (\texttt{c5d.2xlarge}),
72\,GB (\texttt{c5d.9xlarge}), and 144\,GB (\texttt{c5d.18xlarge}).
The training time in each configuration is listed in Table~\ref{table-exp}.

\begin{table}[]
\centering
\label{table-per-tree}
\begin{tabular}{|l|l|c|c|}
\hline
Memory       & \Sparrow         & XGBoost             & LightGBM       \\ \hline
16 GB        & 5.7             & 483.7 (on disk) & (crash)        \\
32 GB        & 5.6             & 255.9 (on disk) & (crash)        \\
72 GB        & 6.8             & 224.7 (on disk) & 16.2          \\
144 GB       & 6.6             & 36.2 (in memory)   & 16.4          \\ \hline
\end{tabular}

\vspace{0.2cm}

\caption{Comparison of the per-tree training time
on the splice site detection task (seconds)}
\end{table}


\subsection{Evaluation}

Performance of each of the algorithm in terms of
the exponential loss as a function of time on the testing dataset is given in
Figure~\ref{fig:loss}. Observe that all algorithms achieve similar
final loss, but it takes them different amount of time to reach that
final loss. We summarize these differences in Table~\ref{table-exp} by
using the convergence time to an almost optimal loss of
$0.061$. Observe  XGBoost off-memory is about 27
times slower than a single \Sparrow\ worker which is also off-memory. That
time improves by another factor of 3.2 by using 10 machines instead of 1.

In Figure~\ref{fig:auprc} we perform the comparison in terms of
AUPRC. The results are similar in terms of speed. However, in this
case XGBoost and LightGBM ultimately achieve a slightly better
AUPRC. This is baffling, because all algorithms work by minimizing
exponential loss.

\begin{figure}[t]
    \centering
    \includegraphics[width=0.5\textwidth]{figs/splice-loss2m.png}
    \caption{Comparing the average loss on the testing data using \Sparrow, XGBoost, and LightGBM, lower is better.
        The period of time that the loss is constant for \Sparrow\ is when the algorithm is generating a new sample set.}~\label{fig:loss}
\end{figure}



\begin{figure}[t]
    \centering
    \includegraphics[width=0.5\textwidth]{figs/splice-auprc2m.png}
    \caption{Comparing the area under the precision-recall curve (AUPRC) on the testing data
    using \Sparrow, XGBoost, and LightGBM, higher is better.
    (left) Normal scale, clipped on right.
    (right) Log scale, clipped on left.
    The period of time that the AUPRC is constant for \Sparrow\ is when the algorithm is generating a new sample set.}~\label{fig:auprc}
\end{figure}


\section{Future Work}\label{sec:Conclusion}
Our preliminary results show that early stopping and selective
sampling can dramatically speed up boosting algorithms on large
real-world datasets.

The source code for sparrow is available for {\tt }

We have several directions for future work.

First, \Sparrow\ is currently limited boosting stumps and to binary
classification. We plan to extend \Sparrow\ to deep trees and to allow
more than two labels. We will then run it on many more data sets.

Second, our work shows that sampling from memory can take an
inordinate amount of time when the training set is large and the
weights are highly skewed. We have developed a stratified sampling
algrithm that significantly reduces this problem.

Third, we are working on a parallelized version of \Sparrow\ which
uses a  novel type of asynchronous communication protocol.

Fourth, our current implementation of \Sparrow\ does not take
advantage of multi-core machines as much as LightGBM does. We plan to
address that problem.







\bibliography{ms}
\bibliographystyle{sysml2019}

\end{document}

